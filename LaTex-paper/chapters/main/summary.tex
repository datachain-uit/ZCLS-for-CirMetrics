\chapter*{\centering{Tóm tắt khóa luận}}
\addcontentsline{toc}{chapter}{Tóm tắt khóa luận}

% Với sự phát triển ngày càng mạnh mẽ của công nghệ blockchain, nhu cầu về mở rộng của hệ thống ngay càng trở thành vẫn đề cấp thiết đối với công nghệ này, bên cạnh vấn đề về bảo mật và phân tán. Khả năng mở rộng vẫn còn là một thách thức lớn đối với các blockchain lớp 1. ZK-Rollup, với việc sử dụng bằng chứng không kiến thức, đã nổi lên như một giải pháp lớp 2 hàng đầu, hứa hẹn cải thiện đáng kể thông lượng giao dịch và giảm chi phí trong khi vẫn kế thừa tính bảo mật của lớp 1. Mặc dù vậy, hiệu suất của các hệ thống ZK-Rollup, đặc biệt là quá trình tạo bằng chứng và biên dịch mạch, vẫn là những rào cản đáng kể đối với việc triển khai thực tế và áp dụng rộng rãi giải pháp này. Các mạch chứng minh không kiến thức phức tạp, đặc biệt là trong các ứng dụng như giao dịch ERC-20, thường chứa một số lượng lớn các ràng buộc, trực tiếp ảnh hưởng đến thời gian tính toán và tài nguyên cần thiết để tạo ra bằng chứng không kiến thức.

% Khoá luận này thực hiện một nghiên cứu thực nghiệm toàn diện nhằm làm sáng tỏ tác động của các tùy chọn tối ưu hóa ràng buộc trong trình biên dịch Circom – một công cụ phổ biến để phát triển mạch chứng minh không kiến thức – đối với hiệu năng của hệ thống ZK-Rollup hỗ trợ giao dịch chuyển token ERC-20 sử dụng hệ thống zk-SNARK Groth16. Cụ thể, nghiên cứu tập trung vào ba cờ tối ưu hóa chính: --O0 (không tối ưu), --O1 (mức tối ưu hóa mặc định), và --O2 (tối ưu hóa mạnh nhất). Thông qua việc mô phỏng một mạch ZK-Rollup cho các giao dịch ERC-20 với các kích thước lô giao dịch đa dạng từ 8 đến 64, nghiên cứu này phân tích định lượng mối quan hệ giữa các mức tối ưu hóa này với số lượng ràng buộc, thời gian biên dịch mạch, và thời gian tạo bằng chứng.

% Kết quả nghiên cứu chỉ ra rằng có một sự đánh đổi rõ rệt giữa thời gian biên dịch và hiệu suất tạo bằng chứng. Cờ --O2, mặc dù loại bỏ hiệu quả gần như toàn bộ ràng buộc tuyến tính (giảm tổng số ràng buộc tới 73.2\% so với --O0 ở batch size 64) và giảm đáng kể thời gian tạo bằng chứng, lại làm tăng đáng kể thời gian biên dịch (lên đến 165\%). Ngược lại, cờ --O1 cung cấp một sự cân bằng hợp lý, giảm khoảng 56\% ràng buộc so với --O0 mà không làm tăng quá nhiều chi phí biên dịch, phù hợp cho giai đoạn phát triển và kiểm thử lặp lại. Nghiên cứu cũng tái khẳng định rằng kích thước bằng chứng và thời gian xác minh bằng chứng trong Groth16 là ổn định, không phụ thuộc vào số lượng ràng buộc, trong khi thời gian tạo bằng chứng chịu ảnh hưởng trực tiếp từ số lượng và bản chất của các ràng buộc, đặc biệt là các ràng buộc phi tuyến phát sinh từ các phép toán mật mã cốt lõi.

% Dựa trên những phân tích thực nghiệm này, khoá luận đề xuất một khung lựa chọn cờ tối ưu hóa thực tiễn, được thiết kế để hỗ trợ các nhà phát triển trong việc đưa ra quyết định dựa trên giai đoạn của dự án (phát triển, kiểm thử, hay sản xuất), tần suất cập nhật mạch, và khối lượng bằng chứng cần tạo. Bằng cách cung cấp dữ liệu thực nghiệm chi tiết và một phương pháp luận có cấu trúc, nghiên cứu này không chỉ làm rõ cơ chế hoạt động của các tùy chọn tối ưu hóa trong Circom mà còn đóng góp vào việc cải thiện quy trình phát triển và triển khai các ứng dụng sử dụng bằng chứng không kiến thức như ZK-Rollup, giúp tối ưu hóa việc sử dụng tài nguyên và đẩy nhanh việc áp dụng công nghệ chứng minh không kiến thức trong các hệ thống blockchain cũng như các ứng dụng sử dụng bằng chứng không kiến thức khác trong thực tế.

% Các kết quả khoa học mà em đã có trong quá trình hoàn thành nội dung khoá luận:
% \begin{itemize}
%     \item Khoá luận sử dụng nội dung nghiên cứu đã được nộp và đang được xem xét tại hội nghị \textbf{ICCAE2026}: Minh Ngo, Khoa Tan-VO, Thu Nguyen, Hong-Tri Nguyen, Tu-Anh Nguyen Hoang: \textit{Improving ZK-Rollup Performance through Constraint Simplification: An Empirical Study Using Circom for ERC-20 transactions}
% \end{itemize}
Khả năng mở rộng vẫn còn là thách thức lớn đối với các blockchain lớp một hiện nay, bên cạnh vấn đề bảo mật và phân tán. ZK-Rollup, dựa trên bằng chứng không kiến thức, là một trong những giải pháp lớp hai tiềm năng nhất, giúp gia tăng thông lượng, giảm chi phí giao dịch mà vẫn giữ nguyên tính bảo mật kế thừa từ blockchain lớp một. Tuy nhiên, hiệu suất của ZK-Rollup vẫn còn là rào cản lớn đối với triển khai thực tế. Các mạch chứng minh trong những ứng dụng phổ biến như giao dịch ERC-20 rất phức tạp, thường chứa một số lượng lớn các ràng buộc, là một trong những nguyên nhân trực tiếp ảnh hưởng đến thời gian tính toán và tài nguyên cần thiết để tạo ra bằng chứng không kiến thức.

Khoá luận này thực hiện nghiên cứu thực nghiệm về tác động của các tùy chọn tối ưu hóa trong Circom -- một công cụ phổ biến cho phát triển mạch SNARK -- đến hiệu năng của hệ thống ZK-Rollup trong giao dịch ERC-20, sử dụng zk-SNARK Groth16. Nghiên cứu tập trung vào ba mức tối ưu: –O0 (không tối ưu), –O1 (mặc định), và –O2 (tối đa). Thông qua mô phỏng mạch với kích thước lô giao dịch từ 8 đến 64, nghiên cứu phân tích định lượng mối quan hệ giữa mức tối ưu hóa với số ràng buộc, thời gian biên dịch và thời gian tạo bằng chứng. Kết quả cho thấy có sự đánh đổi rõ rệt giữa thời gian biên dịch và hiệu quả tạo bằng chứng. –O2 giúp giảm ràng buộc tới 73.2\% so với –O0 (ở batch size 64), rút ngắn đáng kể thời gian tạo bằng chứng nhưng làm tăng thời gian biên dịch lên tới 165\%. Trong khi đó, –O1 cân bằng giữa hiệu quả giảm ràng buộc (giảm \textasciitilde56\% so với --O0) và chi phí biên dịch, phù hợp cho giai đoạn phát triển và kiểm thử lặp lại. Thực nghiệm cũng xác nhận thời gian tạo bằng chứng phụ thuộc trực tiếp vào số lượng và bản chất ràng buộc, đặc biệt các ràng buộc phi tuyến.

Từ đó, khoá luận đề xuất \textbf{ZCLS (ZK-Circuit Lifecycle Strategy)}, một khung hướng dẫn lựa chọn cờ tối ưu hóa theo giai đoạn phát triển. Khung này góp phần hỗ trợ nhà phát triển cân đối tài nguyên, tối ưu hóa quy trình xây dựng ứng dụng ZK-Rollup và các hệ thống sử dụng bằng chứng không kiến thức. 

Các kết quả khoa học mà em đã có trong quá trình hoàn thành nội dung khoá luận:
\begin{itemize}
    % \item Khoá luận sử dụng nội dung nghiên cứu đã được nộp và đang được xem xét tại hội nghị \textbf{ICCAE2026}: Minh Ngo, Khoa Tan-VO, Thu Nguyen, Hong-Tri Nguyen, Tu-Anh Nguyen Hoang: \textit{Improving ZK-Rollup Performance through Constraint Simplification: An Empirical Study Using Circom for ERC-20 transactions}
    \item Khoá luận sử dụng nội dung nghiên cứu đã được nộp và đã được chấp nhận tại hội nghị \textbf{ICCAE2026}: Minh Ngo, Khoa Tan-VO, Thu Nguyen, Hong-Tri Nguyen, Tu-Anh Nguyen Hoang, \textit{"Improving ZK-Rollup Performance through Constraint Simplification: An Empirical Study Using Circom for ERC-20 transactions"}, In The 18th International Conference on Computer and Automation Engineering (ICCAE 2026), 12-14/03/2026, Sydney, Australia.

    \item Đã gửi bài báo tới tạp chí \textbf{IEEE Access} (ngày 13/07/2025) và đang trong quá trình đánh giá: Khoa Vo-Tan, Minh Ngo, Thu Nguyen, Thu-Thuy Ta, Mong-Thy Nguyen-Thi, Hong-Tri Nguyen and Tu-Anh Nguyen-Hoang, “ZCLS: A Lifecycle Strategy for Efficient ZK-Rollup Circuit Optimization in Circom”, IEEE Access (Under Review).
\end{itemize}

