\chapter{Kết luận và hướng phát triển}
\label{chap:chap6}
\section{Kết luận}
Nghiên cứu này đã thực hiện một phân tích thực nghiệm toàn diện về tác động của các cờ tối ưu hóa trình biên dịch Circom (–O0, –O1, –O2) lên hiệu suất của hệ thống ZK-Rollup xử lý giao dịch ERC-20, sử dụng backend Groth16. Thông qua việc mô phỏng và đánh giá các chỉ số quan trọng như số lượng ràng buộc, thời gian biên dịch mạch, và thời gian tạo bằng chứng, nghiên cứ đã làm rõ những đánh đổi hiệu suất liên quan đến từng mức độ tối ưu hóa.

Kết quả cho thấy cờ tối ưu hóa –O2, mặc dù làm tăng đáng kể thời gian biên dịch mạch, lại mang lại hiệu quả vượt trội trong việc giảm số lượng ràng buộc và rút ngắn đáng kể thời gian tạo bằng chứng. Điều này khẳng định giá trị của việc tối ưu hóa sâu ở cấp độ ràng buộc đối với hiệu suất của quá trình tạo bằng chứng, vốn là một nút thắt quan trọng trong các hệ thống ZKP. Ngược lại, cờ –O1 cung cấp một sự cân bằng hợp lý, giảm khoảng 56\% ràng buộc so với –O0 mà không gây ra chi phí biên dịch quá lớn, cho thấy đây là lựa chọn phù hợp cho giai đoạn
phát triển và kiểm thử lặp lại. 

Nghiên cứu cũng tái khẳng định rằng, đối với Groth16, kích thước bằng chứng và thời gian xác minh bằng chứng là ổn định và không phụ thuộc vào số lượng ràng buộc, trong khi thời gian tạo bằng chứng chịu ảnh hưởng trực tiếp từ số lượng và bản chất của các ràng buộc. 

Dựa trên những phân tích định lượng này, nghiên cứu đã đề xuất khung lựa chọn cờ tối ưu hóa thực tiễn \textbf{ZCLS}, giúp các nhà phát triển đưa ra quyết định đúng đắn dựa trên giai đoạn phát triển dự án, tần suất cập nhật mạch và khối lượng bằng chứng cần tạo. Khung này không chỉ tối ưu hóa việc sử dụng tài nguyên mà còn góp phần đẩy nhanh quá trình phát triển và triển khai các ứng dụng sử dụng ZKP thức trong thực tế.

Nghiên cứu này đã cung cấp dữ liệu thực nghiệm chi tiết có giá trị sử dụng làm cơ sở xác thực cho các dự đoán lý thuyết liên quan đến tối ưu hóa hệ ràng buộc, cùng với một phương pháp luận có cấu trúc để hiểu rõ hơn về cơ chế hoạt động của các tùy chọn tối ưu hóa trong Circom. Điều này không chỉ có thể hỗ trợ các nhà phát triển tiết kiệm thời gian và tài nguyên khi phát triển và triển khai các ứng dụng ZK-Rollup mà còn hỗ trợ việc áp dụng rộng rãi công nghệ chứng minh không kiến thức trong các hệ thống blockchain và các ứng dụng khác đòi hỏi tính bảo mật và hiệu suất cao.

Tuy đã nỗ lực để đạt được các mục tiêu nghiên cứu đã đề ra và cung cấp những phân tích định lượng chi tiết về tác động của các cờ tối ưu hóa trong trình biên dịch Circom, nghiên cứu của em vẫn còn một số hạn chế nhất định. Các thử nghiệm được thực hiện trên hệ thống AMD Ryzen 8 nhân, 16 luồng với 32GB RAM, đảm bảo tính nhất quán nhưng có thể cho kết quả khác biệt trên các kiến trúc phần cứng hoặc môi trường phức tạp hơn. Mạch ERC-20 mô phỏng được chọn làm trường hợp nghiên cứu tuy đủ phức tạp để đánh giá tối ưu hóa ràng buộc, nhưng chưa phản ánh đầy đủ sự đa dạng của các hệ thống ZK-Rollup thực tế với nhiều loại giao dịch và hợp đồng thông minh phức tạp. Kết quả thực nghiệm trên các lô nhỏ ($\leq$ 64) không thể dự đoán chính xác hành vi với hàng ngàn giao dịch. Tuy nhiên, kết quả này cung cấp một cái nhìn tổng quan về xu hướng và cơ chế tối ưu hóa, là bước đầu để hiểu hành vi ở quy mô lớn hơn. Ngoài ra, phạm vi nghiên cứu giới hạn ở trình biên dịch Circom và hệ thống zk-SNARK Groth16, chưa so sánh với các hệ thống ZKP khác ví dụ như Plonk. Dù vậy, việc tập trung vào một bộ công cụ cụ thể đã cho phép nghiên cứu phân tích chi tiết hơn, mang lại những hiểu biết quan trọng về tối ưu hóa ràng buộc, đồng thời đây cũng sẽ là nền tảng cho nhiều hướng nghiên cứu tiếp theo.

% Nghiên cứu tập trung chủ yếu vào thời gian biên dịch và tạo bằng chứng, nhưng chưa đi sâu vào phân tích định lượng chi phí bộ nhớ và CPU.

\section{Hướng phát triển}
Để tiếp tục phát triển và mở rộng nghiên cứu này, em đề xuất một số hướng tiềm năng sau:

\begin{itemize}
    \item So sánh với các trình biên dịch ZKP khác: Mở rộng nghiên cứu để so sánh khả năng tối ưu hóa ràng buộc và hiệu suất tổng thể giữa Circom và các trình biên dịch ZKP khác như ZoKrates, Noir, hoặc các framework dựa trên Rust/Halo2. Điều này sẽ cung cấp một cái nhìn toàn diện hơn về bức tranh tối ưu hóa ZKP và giúp các nhà phát triển lựa chọn công cụ phù hợp nhất cho nhu cầu của họ.
    \item Nghiên cứu các kỹ thuật tối ưu hóa mạch nâng cao: Nghiên cứu và thử nghiệm các kỹ thuật tối ưu hóa mạch ZKP ở cấp độ thiết kế mạch, không chỉ giới hạn ở các cờ tối ưu hóa của trình biên dịch. Điều này có thể bao gồm việc sử dụng các cấu trúc dữ liệu hiệu quả hơn, tối ưu hóa các phép toán mật mã cốt lõi, hoặc cải tiến thuật toán tối ưu hoá ràng buộc.
    \item Đánh giá trên các loại giao dịch và ứng dụng khác: Mở rộng phạm vi thực nghiệm để bao gồm các loại giao dịch phức tạp hơn (ví dụ: giao dịch DeFi, NFT minting) hoặc các ứng dụng ZK-Rollup khác ngoài chuyển token ERC-20. Điều này sẽ giúp đánh giá tính tổng quát và khả năng áp dụng của các kết quả tối ưu hóa.
    \item Phân tích tác động của phần cứng: Nghiên cứu sâu hơn về tác động của cấu hình phần cứng (CPU, GPU, bộ nhớ) lên thời gian biên dịch và tạo bằng chứng ở các mức tối ưu hóa khác nhau. Điều này sẽ cung cấp cái nhìn chi tiết hơn về yêu cầu tài nguyên và giúp tối ưu hóa việc triển khai trong môi trường sản xuất.
    \item Phát triển công cụ tự động hóa và tích hợp: Phát triển thêm tính năng cho \textbf{CirMetrics} hỗ trợ để tự động hóa quá trình lựa chọn cờ tối ưu ràng buộc dựa trên các tham số đầu vào của mạch và mục tiêu hiệu suất mong muốn. Mục tiêu là để giúp các nhà phát triển dễ dàng áp dụng các khuyến nghị được đề xuất.
\end{itemize}

Những hướng phát triển này không chỉ giúp làm sâu sắc thêm hiểu biết về tối ưu hóa ZKP mà còn góp phần vào việc xây dựng các hệ thống zk-rollup mạnh mẽ, hiệu quả và dễ dàng hơn trong tương lai.
